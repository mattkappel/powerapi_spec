\section{High Level (Common) Methods}\label{sec:PythonHighLevelCommonMethods}
\subsection{Report Methods}\label{sec:PythonReportMethods}

%==============================================================================%
\subsubsection{Method pwr.Cntxt.GetReportByID} \label{meth:GetReportByID}

Please see the C API specification (section \ref{func:GetReportByID} on page
\pageref{func:GetReportByID}) for a verbose description of the \texttt{GetReportByID()} method. Also, please see
\ref{class:ID} on page \pageref{class:ID} and/or vendor implementation specific
documentation for valid IDs. To collect statistics for an \texttt{idStr} and \texttt{idType} combination, the \texttt{GetReportByID()} method may be used. 

\begin{center}\begin{minipage}{.95\linewidth}\begin{lstlisting}
statValue, timePeriod = myPwrCntxt.GetReportByID(idStr, idType, attrName,
                                                 pwrAttrStat, pwrTimePeriod)
#
# Where:
#   idstr is a string ID for the report to be generated.
#   idType is a pwr.ID type used to interpret the idstr ID
#   attrName is a pwr.AttrName type
#   pwrAttrStat is a pwr.AttrStat object
#   pwrTimePeriod is a pwr.TimePeriod object
# Returns:
#   statValue is the requested statistic
#   timePeriod is the pwr.TimePeriod of the statistic
# This method raises a pwr.PwrError exception when something goes wrong.
# The possible exception errors are: 
#   pwr.ReturnCode.FAILURE
#   pwr.ReturnCode.NOT_IMPLEMENTED
#
\end{lstlisting}\end{minipage}\end{center}


%\subsubsection{Method pwr.Cntxt.GetJobInfo} \label{meth:GetJobInfo}
%\emph{NOTE: this method is proposed by Cray Inc. for inclusion into the specification.  The text represents the APM4 implementation and may be altered pursuant to discussion/review}
%
%This method retrieves attributes, nodes, and timing(s) for a given job and/or run ID. The arguments \texttt{jobID} and \texttt{runID} are keyword arguments, and at least one of \texttt{jobID} or \texttt{runID} must be defined.
%
%\begin{center}\begin{minipage}{.95\linewidth}\begin{lstlisting}
%jobAttr, phys2Nid, jobGrp, jobTP = myPwrCntxt.GetJobInfo(jobID=None, runID=None)
%#
%# Where:
%#   jobID is a string ID for the job (i.e., job ID)
%#   runID is an integer run ID.
%# Returns:
%#   jobAttr is a named tuple of job attributes (job ID, run ID(s), and 
%#       user ID)
%#   physical2NidMap is a dictionary that maps physical component names in 
%#       the job to Node IDs (nids)
%#   jobNodesGrp is a group of comprised of the pwr.ObjType.NODE objects for the
%#       job
%#   timePeriod is a list of pwr.TimePeriod objects for the job. In the case of
%#       a suspended/resumed job, the list has length greater than one.
%# This method raises a pwr.PwrError exception when something goes wrong.
%# The possible exception errors are: 
%#   pwr.ReturnCode.FAILURE
%#   pwr.ReturnCode.NOT_IMPLEMENTED
%#   pwr.ReturnCode.EMPTY
%#   pwr.ReturnCode.BAD_VALUE
%#
%\end{lstlisting}\end{minipage}\end{center}
%
%\subsubsection{Method pwr.Cntxt.GetJobsByUser} \label{meth:GetJobsByUser}
%\emph{NOTE: this method is proposed by Cray Inc. for inclusion into the specification.  The text represents the APM4 implementation and may be altered pursuant to discussion/review}
%
%This method of the \texttt{pwr.Cntxt} class gets a list of (job ID, run ID) tuples for a given userId (uid) and time period.
%
%\begin{center}\begin{minipage}{.95\linewidth}\begin{lstlisting}
%jobList = myPwrCntxt.GetJobsByUser(userId, pwrTimePeriod=None)
%#
%# Where:
%#   userId is a user ID (integer uid)
%#   pwrTimePeriod is a pwr.TimePeriod object to scope the query. If not supplied,
%#       the entire job history is searched.
%# Returns:
%#   A list of (job ID, apid) tuples matching the user ID userId and the time
%#       period pwrTimePeriod (if supplied).
%# This method raises a pwr.PwrError exception when something goes wrong.
%# The possible exception errors are: 
%#   pwr.ReturnCode.FAILURE
%#   pwr.ReturnCode.NOT_IMPLEMENTED
%#   pwr.ReturnCode.BAD_VALUE
%#
%\end{lstlisting}\end{minipage}\end{center}
%
%\subsubsection{Method pwr.Cntxt.GetJobsByTime} \label{meth:GetJobsByTime}
%\emph{NOTE: this method is proposed by Cray Inc. for inclusion into the specification.  The text represents the APM4 implementation and may be altered pursuant to discussion/review}
%
%This method of the \texttt{pwr.Cntxt} class gets a list of (job ID, runID) tuples for a given (optional) time period.
%
%\begin{center}\begin{minipage}{.95\linewidth}\begin{lstlisting}
%jobList = myPwrCntxt.GetJobsByTime(pwrTimePeriod=None)
%#
%# Where:
%#   pwrTimePeriod is a pwr.TimePeriod object to scope the query. If not supplied,
%#       the entire job history is searched.
%# Returns:
%#   A list of (job ID, apid) tuples matching the time period pwrTimePeriod 
%#       (if supplied).
%# This method raises a pwr.PwrError exception when something goes wrong.
%# The possible exception errors are: 
%#   pwr.ReturnCode.FAILURE
%#   pwr.ReturnCode.NOT_IMPLEMENTED
%#
%\end{lstlisting}\end{minipage}\end{center}
%
%\subsubsection{Method pwr.Obj.WriteTimeSeriesCSV} \label{meth:objWriteTimeSeriesCSV}
%\emph{NOTE: this method is proposed by Cray Inc. for inclusion into the specification.  The text represents the APM4 implementation and may be altered pursuant to discussion/review}
%
%This method of the \texttt{pwr.Obj} class writes a time series for an object to a file as comma-separated values (CSV). The column order of the CSV file is: timestamp, object name, attribute name, value.
%
%\begin{center}\begin{minipage}{.95\linewidth}\begin{lstlisting}
%rc = myPwrObj.WriteTimeSeriesCSV(pwrAttr, pwrTimePeriod, filename)
%#
%# Where:
%#   pwrAttr is a Power API attribute (e.g., pwr.AttrName.POWER, 
%#       pwr.AttrName.ENERGY)
%#   pwrTimePeriod is a list of one or more non-overlapping pwr.TimePeriod 
%#       objects under which to generate a time series. 
%#   filename is the path and filename to which to write the comma-separated 
%#       values. A filename of `stdout' is treated as a special, using 
%#       sys.stdout to print to stdout.
%# Returns:
%#   A return code, pwr.ReturnCode.SUCCESS for success.
%# This method raises a pwr.PwrError exception when something goes wrong.
%# The possible exception errors are: 
%#   pwr.ReturnCode.FAILURE
%#   pwr.ReturnCode.NOT_IMPLEMENTED
%#   pwr.ReturnCode.BAD_VALUE
%#
%\end{lstlisting}\end{minipage}\end{center}
%
%
%\subsubsection{Method pwr.Grp.WriteTimeSeriesCSV} \label{meth:grpWriteTimeSeriesCSV}
%\emph{NOTE: this method is proposed by Cray Inc. for inclusion into the specification.  The text represents the APM4 implementation and may be altered pursuant to discussion/review}
%
%This method of the \texttt{pwr.Grp} class writes a time series for each object in a group to a file as comma-separated values (CSV). The column order of the CSV file is: timestamp, object name, attribute name, value.
%
%\begin{center}\begin{minipage}{.95\linewidth}\begin{lstlisting}
%rc = myPwrGrp.WriteTimeSeriesCSV(pwrAttr, pwrTimePeriod, filename)
%#
%# Where:
%#   pwrAttr is a Power API attribute (e.g., pwr.AttrName.POWER, 
%#       pwr.AttrName.ENERGY)
%#   pwrTimePeriod is a list of one or more non-overlapping pwr.TimePeriod 
%#       objects under which to generate a time series. 
%#   filename is the path and filename to which to write the comma-separated 
%#       values. A filename of `stdout' is treated as a special, using 
%#       sys.stdout to print to stdout.
%# Returns:
%#   A return code, pwr.ReturnCode.SUCCESS for success.
%# This method raises a pwr.PwrError exception when something goes wrong.
%# The possible exception errors are: 
%#   pwr.ReturnCode.FAILURE
%#   pwr.ReturnCode.NOT_IMPLEMENTED
%#   pwr.ReturnCode.BAD_VALUE
%#   pwr.ReturnCode.OP_NOT_ATTEMPTED
%#
%\end{lstlisting}\end{minipage}\end{center}
%
%\subsubsection{Method pwr.Grp.WriteAggregateTimeSeriesCSV} \label{meth:WriteAggregateTimeSeriesCSV}
%\emph{NOTE: this method is proposed by Cray Inc. for inclusion into the specification.  The text represents the APM4 implementation and may be altered pursuant to discussion/review}
%
%This method of the \texttt{pwr.Grp} class writes an aggregated time series for a group of objects to a file as comma-separated values (CSV). The column order of the CSV file is: timestamp, attribute name, value.  This call is useful for getting a time series of a total over a group (e.g., total power over a group of nodes).
%
%\begin{center}\begin{minipage}{.95\linewidth}\begin{lstlisting}
%rc = myPwrGrp.WriteAggregateTimeSeriesCSV(pwrAttr, pwrTimePeriod, filename)
%#
%# Where:
%#   pwrAttr is a Power API attribute (e.g., pwr.AttrName.POWER, 
%#       pwr.AttrName.ENERGY)
%#   pwrTimePeriod is a list of one or more non-overlapping pwr.TimePeriod 
%#       objects under which to generate a time series. 
%#   filename is the path and filename to which to write the comma-separated 
%#       values. A filename of `stdout' is treated as a special, using 
%#       sys.stdout to print to stdout.
%# Returns:
%#   A return code, pwr.ReturnCode.SUCCESS for success.
%# This method raises a pwr.PwrError exception when something goes wrong.
%# The possible exception errors are: 
%#   pwr.ReturnCode.FAILURE
%#   pwr.ReturnCode.NOT_IMPLEMENTED
%#   pwr.ReturnCode.BAD_VALUE
%#   pwr.ReturnCode.OP_NOT_ATTEMPTED
%#
%\end{lstlisting}\end{minipage}\end{center}
%
%%==============================================================================%
%\subsection{Convenience Methods}\label{sec:PythonConvenienceMethods}
%
%\subsubsection{Method pwr.Cntxt.GrpCreateByJobID} \label{meth:GrpCreateByJobID}
%\emph{NOTE: this method is proposed by Cray Inc. for inclusion into the specification.  The text represents the APM4 implementation and may be altered pursuant to discussion/review}
%
%Given a Job ID \texttt{jobIDStr}, this method of the \texttt{pwr.Cntxt} class returns a group of objects (of type \texttt{objType}) associated with the job.
%
%\begin{center}\begin{minipage}{.95\linewidth}\begin{lstlisting}
%jobGroup = myPwrCntxt.GrpCreateByJobID(jobIDStr, objType)
%#
%# Where:
%#   jobIDStr is a string ID for the job (i.e., job ID)
%#   objType is a Power API object type (e.g., pwr.ObjType.NODE)
%#      
%# Returns:
%#   A group of objects of type objType associated with job ID jobIDStr.
%# This method raises a pwr.PwrError exception when something goes wrong.
%# The possible exception errors are: 
%#   pwr.ReturnCode.FAILURE
%#   pwr.ReturnCode.NOT_IMPLEMENTED
%#   pwr.ReturnCode.BAD_VALUE
%#
%\end{lstlisting}\end{minipage}\end{center}
%
%\subsubsection{Method pwr.Cntxt.GrpCreateByRunID} \label{meth:GrpCreateByRunID}
%\emph{NOTE: this method is proposed by Cray Inc. for inclusion into the specification.  The text represents the APM4 implementation and may be altered pursuant to discussion/review}
%
%Given a Run ID \texttt{runID}, this method of the \texttt{pwr.Cntxt} class returns a group of objects (of type \texttt{objType}) associated with the run.
%
%\begin{center}\begin{minipage}{.95\linewidth}\begin{lstlisting}
%runGroup = myPwrCntxt.GrpCreateByRunID(runID, objType)
%#
%# Where:
%#   runID is an integer ID for the run (i.e., run ID)
%#   objType is a Power API object type (e.g., pwr.ObjType.NODE)
%#      
%# Returns:
%#   A group of objects of type objType associated with run ID runID.
%# This method raises a pwr.PwrError exception when something goes wrong.
%# The possible exception errors are: 
%#   pwr.ReturnCode.FAILURE
%#   pwr.ReturnCode.NOT_IMPLEMENTED
%#   pwr.ReturnCode.BAD_VALUE
%\end{lstlisting}\end{minipage}\end{center}
%
%\subsubsection{Method pwr.Cntxt.GrpCreateByName} \label{meth:GrpCreateByName}
%\emph{NOTE: this method is proposed by Cray Inc. for inclusion into the specification.  The text represents the APM4 implementation and may be altered pursuant to discussion/review}
%
%This method of the \texttt{pwr.Cntxt} class creates a group given a tuple of object/component physical names. If bad or non-existent names are given, they are silently ignored.
%
%\begin{center}\begin{minipage}{.95\linewidth}\begin{lstlisting}
%physNameGroup = myPwrCntxt.GrpCreateByName(nameTuple, objType)
%#
%# Where:
%#   nameTuple is a a tuple of string object/component physical names.
%#   objType is a Power API object type (e.g., pwr.ObjType.NODE)
%#      
%# Returns:
%#   A group of objects derived from nameTuple.
%# This method raises a pwr.PwrError exception when something goes wrong.
%# The possible exception errors are: 
%#   pwr.ReturnCode.FAILURE
%#   pwr.ReturnCode.NOT_IMPLEMENTED
%#   pwr.ReturnCode.BAD_VALUE
%\end{lstlisting}\end{minipage}\end{center}
%
%\subsubsection{Method pwr.Cntxt.GrpCreateByNidlist} \label{meth:GrpCreateByNidlist}
%\emph{NOTE: this method is proposed by Cray Inc. for inclusion into the specification.  The text represents the APM4 implementation and may be altered pursuant to discussion/review}
%
%This method of the \texttt{pwr.Cntxt} class creates a group given a tuple of node IDs (nids). If bad or non-existent nids are given, they are silently ignored.
%
%\begin{center}\begin{minipage}{.95\linewidth}\begin{lstlisting}
%nidlistGroup = myPwrCntxt.GrpCreateByNidlist(nidTuple, objType)
%#
%# Where:
%#   nidTuple is a a tuple of integer node IDs (nids).
%#   objType is a Power API object type (e.g., pwr.ObjType.NODE)
%#      
%# Returns:
%#   A group of objects derived from nidTuple.
%# This method raises a pwr.PwrError exception when something goes wrong.
%# The possible exception errors are: 
%#   pwr.ReturnCode.FAILURE
%#   pwr.ReturnCode.NOT_IMPLEMENTED
%#   pwr.ReturnCode.BAD_VALUE
%\end{lstlisting}\end{minipage}\end{center}
