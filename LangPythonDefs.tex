\section{Type Definitions}\label{sec:PythonTypeDefinitions}

This chapter lists the enumerations and classes associated with the Python
version of the Power API and mirrors the naming and numbering used in the C
API specification (found in chapter \ref{chap:TypeDefinitions} starting on page
\pageref{chap:TypeDefinitions}). The enumerations listed in this section are
required to exist, but not all enumerated values are required to be supported
by any specific Python implementation of the Power API. Some of the
enumerations are meant to be expanded, while some are not. Each of the sections
below discuss what compliant Python versions of these enumerations and
structures (classes) look like and whether they can be expanded upon. 

\subsection{Opaque Types}\label{sec:PythonOpaqueTypes}
The opaque types described in \ref{sec:OpaqueTypes} on page
\pageref{sec:OpaqueTypes} are represented as Python base classes with the
exception of the \texttt{PWR_Status} structure. The \texttt{PWR_Status}
structure is used for returning error status on functions that perform multiple
operations.  Python implementations handle these errors differently and do not
need to use the \texttt{PWR_Status} structure.

\begin{center}\begin{minipage}{.95\linewidth}\begin{lstlisting}
class Cntxt(...):
    ...
class Grp(...):
    ...
class Obj(...):
    ...
class Stat(...):
    ...
\end{lstlisting}\end{minipage}\end{center}

These opaque abstract classes are meant to be overloaded by specific
implementations of this Python API. The discussion of the object type child
classes of the \texttt{pwr.Obj} class in section
\ref{sec:PythonSystemDescription} on page \pageref{sec:PythonSystemDescription}
illustrates this. 

\subsection{Globally Relevant Definitions}
\label{sec:PythonGloballyRelevantDefinitions} 

The Python bindings support all of the C API's global definitions (see page
\pageref{sec:GlobalTypes}), such as the version number definitions that are
useful in the Python API. Definitions like the maximum length of text-strings,
are not useful since Python automatically handles allocation and garbage
collection for strings. PWR_MAJOR_VERSION and PWR_MINOR_VERSION are exposed
through the \texttt{pwr.GetMajorVersion()} and \texttt{pwr.GetMinorVersion()}
functions defined in section \ref{sec:PythonVersionFunctions} on page
\pageref{sec:PythonVersionFunctions}).

\subsection{Context Relevant Type Definitions}
\label{sec:PythonContextReleventTypeDefinitions}

The Python bindings support the necessary power contexts needed for 
implementation and follows the design of the C API as defined on page
\pageref{sec:ContextTypeDefinitions}. Power contexts use the Python
enumeration scheme described in \ref{sec:Enumerations} on starting on page
\pageref{sec:Enumerations} and contain a single ``Default'' power context
enumeration. The default context type carries with it the default capabilities
of the API. For vendor, platform, and model-specific capabilities, implementors can add new context types.

%==============================================================================%
\subsubsection{Enumeration Class CntxtType}\label{class:CntxtType}

All implementations must support the \texttt{"DEFAULT"} context. The corresponding C
API enumeration is on page \pageref{type:CntxtType}. The following is the enumeration
for the default context type:

\begin{center}\begin{minipage}{.95\linewidth}\begin{lstlisting}
class CntxtType(_EnumerationClass):
    pass
CntxtType("DEFAULT")
\end{lstlisting}\end{minipage}\end{center}

Add the following to extend the enumeration for context types for a new, non-default vendor.
(see ``Enumerations Extension'' \ref{sec:EnumerationsExtension} on page
\pageref{sec:EnumerationsExtension}):

\begin{center}\begin{minipage}{.95\linewidth}\begin{lstlisting}
CntxtType("VENDORNAME")
\end{lstlisting}\end{minipage}\end{center}

%==============================================================================%
\subsubsection{Enumeration Class Role}\label{class:Role}

Default roles are defined by the \texttt{pwr.Role} enumeration. All
contexts support one or more of these roles. See page
\pageref{type:Role} for the C API \texttt{enum} definition.

\begin{center}\begin{minipage}{.95\linewidth}\begin{lstlisting}
class Role(_EnumerationClass):
    pass
Role("APP")    # Application
Role("MC")     # Monitor and Control
Role("OS")     # Operating System
Role("USER")   # User
Role("RM")     # Resource Manager
Role("ADMIN")  # Administrator
Role("MGR")    # HPCS Manager
Role("ACC")    # Accounting
#
# Vendor implementations SHALL NOT add roles!
#
Role("NUM_ROLES")
Role("INVALID", -1)
Role("NOT_SPECIFIED", -2)
\end{lstlisting}\end{minipage}\end{center}\label{sec:PythonPWRRoleDefinition}

\subsection{Object Relevant Type Definitions}
\label{sec:PythonObjectRelevantTypeDefinitions}

%==============================================================================%
\subsubsection{Enumeration Class ObjType}\label{class:ObjType}

All implementations of the Power API are required to have the following object
types enumerated. Implementations may add object types to these defaults, but
must do so using the methods described in \ref{sec:EnumerationsExtension} on
page \pageref{sec:EnumerationsExtension}. The corresponding C API enumeration
is on page \pageref{type:ObjType}:

\begin{center}\begin{minipage}{.95\linewidth}\begin{lstlisting}
class ObjType(_EnumerationClass):
    pass
ObjType("PLATFORM")
ObjType("CABINET")
ObjType("CHASSIS")
ObjType("BOARD")
ObjType("NODE")
ObjType("SOCKET")
ObjType("CORE")
ObjType("POWER_PLANE")
ObjType("MEM")
ObjType("NIC")
# Vendor implementations can add more entries here or look at  
# the object-oriented "Enumerations Extension" description previously. 
ObjType("NUM_OBJ_TYPES")
ObjType("INVALID", -1)
ObjType("NOT_SPECIFIED", -2)
\end{lstlisting}\end{minipage}\end{center}

\subsection{Attribute Relevant Type Definitions}
\label{sec:PythonAttributeRelevantTypeDefinitions}

%==============================================================================%
\subsubsection{Enumeration Class AttrName}\label{class:AttrName}

The following default attributes must be enumerated in any implementation of
this API. If more attributes are desired to be added for a particular
implementation, see the methods described in \ref{sec:EnumerationsExtension} on
page \pageref{sec:EnumerationsExtension}. The corresponding C API enumeration
is on page \pageref{type:AttrName}:

\begin{center}\begin{minipage}{.95\linewidth}\begin{lstlisting}
class AttrName(_EnumerationClass):
    pass
AttrName("PSTATE")           # Python long
AttrName("CSTATE")           # Python long
AttrName("CSTATE_LIMIT")     # Python long
AttrName("SSTATE")           # Python long
AttrName("CURRENT")          # Python float, amps
AttrName("VOLTAGE")          # Python float, volts
AttrName("POWER")            # Python float, watts
AttrName("POWER_LIMIT_MIN")  # Python float, watts
AttrName("POWER_LIMIT_MAX")  # Python float, watts
AttrName("FREQ")             # Python float, Hz
AttrName("FREQ_LIMIT_MIN")   # Python float, Hz
AttrName("FREQ_LIMIT_MAX")   # Python float, Hz
AttrName("ENERGY")           # Python float, joules 
AttrName("TEMP")             # Python float, degrees Celsius
AttrName("OS_ID")            # Python long
AttrName("THROTTLED_TIME")   # Python long
AttrName("THROTTLED_COUNT")  # Python long
# Vendor implementations can add more entries here or look at  
# the object-oriented "Enumerations Extension" description previously. 
AttrName("NUM_ATTR_NAMES")
AttrName("INVALID", -1)
AttrName("NOT_SPECIFIED", -2)
\end{lstlisting}\end{minipage}\end{center}

%==============================================================================%
\subsubsection{Built-in Support for AttrDataType}
\label{sec:PythonPwrAttrDataType}

In the C API specification a \texttt{pwr.AttrDataType} enumeration is defined.
This is to ease the type checking of data coming from various power attributes.
In Python, object ``typing'' is built-in such that this enumeration
becomes redundant and not meaningful.

There are two basic data types found to represent the various values that may
be used with any valid Python API methods; ``long'' and ``float''. They may be
type-checked as follows:

\begin{center}\begin{minipage}{.95\linewidth}\begin{lstlisting}
val = SomeMethodOrFunction()
if not isinstance(val, float):
    raise pwr.PwrError(pwr.ReturnCode.BAD_VALUE, "Bad temperature value returned!")
\end{lstlisting}\end{minipage}\end{center}

%==============================================================================%
\subsubsection{Class AttrAccessError} \label{class:AttrAccessError}

In the Python API, the \texttt{PWR_AttrAccessError} (along with the
\texttt{PWR_Stat} structure), have been replaced by the functionality of
\textit{measurement} named tuples and lists of \textit{measurement} named tuples. For
details on these named tuples, please refer to the discussion on \texttt{AttrGetValue}
and \texttt{AttrSetValue} starting at \ref{meth:ObjAttrGetValue} on page
\pageref{meth:ObjAttrGetValue}.

\subsection{Metadata Relevant Type Definitions}
\label{sec:PythonMetadataRelevantTypeDefinitions}

%==============================================================================%
\subsubsection{Enumeration Class MetaName}\label{class:MetaName}

The default implementation/context must at least have these Metadata names
enumerated. Additional metadata names may be defined using the methods
described in \ref{sec:EnumerationsExtension} on page
\pageref{sec:EnumerationsExtension}. The corresponding C API enumeration is on
page \pageref{type:MetaName}:


\begin{center}\begin{minipage}{.95\linewidth}\begin{lstlisting}
class MetaName(_EnumerationClass):
    pass
MetaName("NUM")              # Python long
MetaName("MIN")              # Python long or Float (depending on attr. type)
MetaName("MAX")              # Python long or Float (depending on attr. type)
MetaName("PRECISION")        # Python long
MetaName("ACCURACY")         # Python Float
MetaName("UPDATE_RATE")      # Python Float
MetaName("SAMPLE_RATE")      # Python Float
MetaName("TIME_WINDOW")      # pwr.Time object
MetaName("TS_LATENCY")       # pwr.Time object
MetaName("TS_ACCURACY")      # pwr.Time object
MetaName("MAX_LEN")          # Python long (max length of any metadata string)
MetaName("NAME_LEN")         # Python long (max length of NAME)
MetaName("NAME")             # Python String
MetaName("DESC_LEN")         # Python long (max length of DESC)
MetaName("DESC")             # Python String
MetaName("VALUE_LEN")        # Python long (max length of meta value at index)
MetaName("VENDOR_INFO_LEN")  # Python long (max length of VENDOR_INFO
MetaName("VENDOR_INFO")      # Python String
MetaName("MEASURE_METHOD")   # Python long (0/1 depending on real/model meas.)
# Vendor implementations can add more entries here or look at  
# the object-oriented "Enumerations Extension" description previously. 
MetaName("NUM_META_NAMES")
MetaName("INVALID", -1)
MetaName("NOT_SPECIFIED", -2)
\end{lstlisting}\end{minipage}\end{center}

\emph{Implementation Note: The ``LEN''-related definitions above are not
useful in any Python implementation but are included for consistency with the C API specification.}

\subsection{Error Return Definitions}\label{sec:PythonErrorReturnDefinition}

%==============================================================================%
\subsubsection{Enumeration Class ReturnCode}\label{class:ReturnCode}

The following error definitions are required to be defined for every
implementation of the API. New return code definitions may be added at the end
of this list. The corresponding C API enumeration is on page
\pageref{sec:ErrorReturnDefinitions}.

\begin{center}\begin{minipage}{.95\linewidth}\begin{lstlisting}
class ReturnCode(_EnumerationClass):
    negative = True
ReturnCode("WARN_NO_GRP_BY_NAME",  5)
ReturnCode("WARN_NO_OBJ_BY_NAME",  4)
ReturnCode("WARN_NO_CHILDREN",     3)
ReturnCode("WARN_NO_PARENT",       2)
ReturnCode("WARN_NOT_OPTIMIZED",   1)
#
ReturnCode("SUCCESS")            #  0
ReturnCode("FAILURE")            # -1
ReturnCode("NOT_IMPLEMENTED")    # -2
ReturnCode("EMPTY")              # -3
ReturnCode("INVALID")            # -4
ReturnCode("LENGTH")             # -5
ReturnCode("NO_ATTRIB")          # -6
ReturnCode("NO_META")            # -7
ReturnCode("READ_ONLY")          # -8
ReturnCode("BAD_VALUE")          # -9
ReturnCode("BAD_INDEX")          # -10
ReturnCode("OPT_NOT_ATTEMPTED")  # -11
ReturnCode("NO_PERM")            # -12
ReturnCode("OUT_OF_RANGE")       # -13
ReturnCode("NO_OBJ_AT_INDEX")    # -14
\end{lstlisting}\end{minipage}\end{center}

\subsection{Time Related Definitions}
\label{sec:PythonTimeRelatedDefinitions}

In the C API, \texttt{uint64} (unsigned 64-bit integer) values are used to represent a
value in nanoseconds. Native Python time values are stored in floating point
format. A full description of why Python implementations of the Power API needs
some extra features is given in section \ref{sec:PythonTimeEntities} on page
\pageref{sec:PythonTimeEntities}. The preferred implementation for handling
time in Python implementations of the Power API is as follows:

%==============================================================================%
\subsubsection{Class Time} \label{class:Time}

\begin{center}\begin{minipage}{.95\linewidth}\begin{lstlisting}
class Time(long):
    def __init__(self, timeVal):
        # convert to ns if timeVal is a float (seconds) value
        ...
# To access:
now = pwr.Time():    # A generic, opaque time value
nowNs = long(now)    # Converts and returns the time value to a long integer
                     #   representing a number of nanoseconds.
nowSec = float(now)  # Converts and returns the time value to a floating point
                     # value representing a number of seconds.
\end{lstlisting}\end{minipage}\end{center}

A set of definitions is used for defining what a time value set to
\texttt{None} means:

\begin{center}\begin{minipage}{.95\linewidth}\begin{lstlisting}
PWR_TIME_UNINIT  = None  # Time value was never initialized
PWR_TIME_UNKNOWN = None  # Time value was never recorded
\end{lstlisting}\end{minipage}\end{center}

%==============================================================================%
\subsubsection{Class TimePeriod} \label{class:TimePeriod}

The \texttt{PWR_TimePeriod} struct in C on page \pageref{type:TimePeriod}, is
represented in Python by a class:

\begin{center}\begin{minipage}{.95\linewidth}\begin{lstlisting}

class TimePeriod():
    def __init__(self,
                 start = TIME_UNINIT,
                 stop = TIME_UNINIT,
                 instant = TIME_UNINIT):
        self._start = Time(start)
        self._stop  = Time(stop)
        self._instant = Time(instant)
    ...
\end{lstlisting}\end{minipage}\end{center}

To access the various data items of the pwr.TimePeriod class instance,
``myTimePeriod'': 

\begin{center}\begin{minipage}{.95\linewidth}\begin{lstlisting}
# Get (read) access
startTimeSec = float(myTimePeriod.start)  # as Seconds     (float value)
stopTimeNs = long(myTimePeriod.stop)      # as Nanoseconds (long value)
instantTime = myTimePeriod.instant        # as pwr.Time    (pwr.Time object)
# Set (write) access
myTimePeriod.start = startTimeNs          # (a long value)
myTimePeriod.stop = stopTimeSec           # (a float value)
myTimePeriod.instant = PWR_TIME_UNINIT    # (None)
\end{lstlisting}\end{minipage}\end{center}

\subsection{Statistics Relevant Type Definitions}
\label{sec:PythonStatisticsRelevantTypeDefinitions}

For background on the overall support for statistics in the Power API refer to
section \ref{sec:StatisticTypeDefinitions} on page
\pageref{sec:StatisticTypeDefinitions}, and  the ``Statistics Functions'' in
section \ref{sec:StatisticsFunctions} starting on page
\pageref{sec:StatisticsFunctions}.

%==============================================================================%
\subsubsection{Enumeration Class AttrStat}\label{class:AttrStat}

The following class \texttt{AttrStat(_EnumerationClass)} includes the list of
currently-defined statistics potentially available to the user of an
implementation. Additional Statistics operations may be vendor-defined using
the methods described in \ref{sec:EnumerationsExtension} on page
\pageref{sec:EnumerationsExtension}. See section \ref{type:AttrStat} on page
\pageref{type:AttrStat} for the C API enumeration:

\begin{center}\begin{minipage}{.95\linewidth}\begin{lstlisting}
class AttrStat(_EnumerationClass):
    pass
AttrStat("MIN")
AttrStat("MAX")
AttrStat("AVG")
AttrStat("STDEV")
AttrStat("CV")
AttrStat("SUM")
# Vendor implementations can add more entries here or look at  
# the object-oriented "Enumerations Extension" description previously. 
AttrStat("NUM_ATTR_STATS")
AttrStat("INVALID", -1)
AttrStat("NOT_SPECIFIED", -2)
\end{lstlisting}\end{minipage}\end{center}

%==============================================================================%
\subsubsection{Enumeration Class ID}\label{class:ID}

The C API definition for the PWR_ID enumeration is on page \pageref{type:ID}.
Python implementations use ID-enumerated types in support of the method
GetReportByID in section \ref{meth:GetReportByID}. Vendor specific additions to
this enumeration class can be added using the methods described in
\ref{sec:EnumerationsExtension} on page \pageref{sec:EnumerationsExtension}.
The corresponding C API enumeration is on page \pageref{type:ID}:

\begin{center}\begin{minipage}{.95\linewidth}\begin{lstlisting}
class ID(_EnumerationClass):
    pass
ID("USER")
ID("JOB")
ID("RUN")
#
ID("NUM_IDS")
ID("INVALID", -1)
ID("NOT_SPECIFIED", -2)
\end{lstlisting}\end{minipage}\end{center}

\subsection{OS Hardware Type Definitions}\label{sec:PythonOSHardwareTypeDefinitions}

%==============================================================================%
\subsubsection{Class OperState} \label{class:OperState}

A Python class represents the C API \texttt{PWR_OperState} structure as follows:

\begin{center}\begin{minipage}{.95\linewidth}\begin{lstlisting}
class OperState():
    def __init__(self, cStateNum, pStateNum):
        self.c_state_num = cStateNum
        self.p_state_num = pStateNum
    ...
\end{lstlisting}\end{minipage}\end{center}

To access the various data items of the \texttt{pwr.OperState} class instance,
\texttt{myOpState}: 

\begin{center}\begin{minipage}{.95\linewidth}\begin{lstlisting}
# Get (read) access
cState = myOpState.c_state_num
pState = myOpState.p_state_num
# Set (write) access
myOpState.c_state_num = pwr.SleepState.SHALLOW
myOpState.p_state_num = pwr.PerfState.FASTEST

\end{lstlisting}\end{minipage}\end{center}

\subsection{Application OS Interface Type Definitions}
\label{sec:PythonApplicationOSInterfaceTypeDefinitions}

%==============================================================================%
\subsubsection{Enumeration Class RegionHint}\label{class:RegionHint}

Please see page \pageref{type:RegionHint} for the C API description of this
enumeration.

\begin{center}\begin{minipage}{.95\linewidth}\begin{lstlisting}
class RegionHint(_EnumerationClass):
    pass
RegionHint("DEFAULT")
RegionHint("SERIAL")
RegionHint("PARALLEL")
RegionHint("COMPUTE")
RegionHint("COMMUNICATE")
RegionHint("IO")
RegionHint("MEM_BOUND")
# Vendor implementations can add more entries here or look at  
# the object-oriented "Enumerations Extension" description previously. 
RegionHint("NUM_REGION_HINTS")
RegionHint("INVALID", -1)
RegionHint("NOT_SPECIFIED", -2)
\end{lstlisting}\end{minipage}\end{center}

%==============================================================================%
\subsubsection{Enumeration Class RegionIntensity}\label{class:RegionIntensity}

Please see page \pageref{type:RegionIntensity} for the C API description of this
enumeration.

\begin{center}\begin{minipage}{.95\linewidth}\begin{lstlisting}
class RegionIntensity(_EnumerationClass):
    pass
RegionIntensity("HIGHEST")
RegionIntensity("HIGH")
RegionIntensity("MEDIUM")
RegionIntensity("LOW")
RegionIntensity("LOWEST")
RegionIntensity("NONE")
# Vendor implementations can add more entries here or look at  
# the object-oriented "Enumerations Extension" description previously. 
RegionIntensity("NUM_REGION_INTENSITIES")
RegionIntensity("INVALID", -1)
RegionIntensity("NOT_SPECIFIED", -2)
\end{lstlisting}\end{minipage}\end{center}

%==============================================================================%
\subsubsection{Enumeration Class SleepState}\label{class:SleepState}

Please see page \pageref{type:SleepState} for the C API description of this
enumeration.

\begin{center}\begin{minipage}{.95\linewidth}\begin{lstlisting}
class SleepState(_EnumerationClass):
    pass
SleepState("NO")
SleepState("SHALLOW")
SleepState("MEDIUM")
SleepState("DEEP")
SleepState("DEEPEST")
# Vendor implementations can add more entries here or look at  
# the object-oriented "Enumerations Extension" description previously. 
SleepState("NUM_SLEEP_STATES")
SleepState("INVALID", -1)
SleepState("NOT_SPECIFIED", -2)
\end{lstlisting}\end{minipage}\end{center}

%==============================================================================%
\subsubsection{Enumeration Class PerfState}\label{class:PerfState}

Please see page \pageref{type:PerfState} for the C API description of this
enumeration.

\begin{center}\begin{minipage}{.95\linewidth}\begin{lstlisting}
class PerfState(_EnumerationClass):
    pass
PerfState("FASTEST")
PerfState("FAST")
PerfState("MEDIUM")
PerfState("SLOW")
PerfState("SLOWEST")
# Vendor implementations can add more entries here or look at  
# the object-oriented "Enumerations Extension" description previously. 
PerfState("NUM_PERF_STATES")
PerfState("INVALID", -1)
PerfState("NOT_SPECIFIED", -2)
\end{lstlisting}\end{minipage}\end{center}


